\newcommand{\projectName}{Az EGTIB}

\chapter{\projectName{} bemutatása}

\projectName{} projekt célja egy olyan felhasználóbarát felület létrehozása, mely lehetőséget teremt a daganatos sejtek játékelméleti modellezésére és szimulációjára, a szimulációs eredmények megjelenítésére illetve ezeknek valamilyen szintű mentésére is. Ezért született meg, a mai trendeket figyelembe véve, egy kliens-szerver architektúrán alapuló webalkalmazás, mely részben az \cite{archetti2016cooperation}-ben megjelent modellt implementálja, és ahhoz új funkcionalitásokat is hozzáad.

\begin{figure}[ht!]
	\centering
	\includegraphics[width=90mm]{images/EGTIB.jpg}
	\caption{Pillanatkép az alkalmazásról \label{fig:SimulateWithDiagram}}
\end{figure}

\section{Funkcionalitások}

Az alkalmazásunkat, funkcionalitás szerint, három fő komponensre oszthatjuk:

\paragraph{Vizualizáció}- képes szimulálni kevés sejtből (max. 500) álló populációkat és ezeket generációnként meg is tudja jeleníteni. A megjelenítés interaktív, ami azt jelenti, hogy bármikor meg lehet állítani, folytatni vagy akár tekergetni mint egy filmet. Továbbá mindehhez tartozik egy grafikon is, mely a populációnak változását ábrázolja generációnként.

\paragraph{Szimuláció}- alkalmas nagyobb (max. 2000) populációval dolgozni, viszont ezek megjelenítése már túl sok időt venne igénybe, így csak a fentebb említett grafikon segítségével ábrázolja azt, hogy mi is történik a sejtek között. Előnye a vizualizációs részhez képest elsősorban az, hogy sokkal nagyobb populációval tud dolgozni, de a paraméterlista is változatosabb (a szimuláció során használt függvények viselkedésébe is bele tud szólni a felhasználó) és hatalmas kényelmi faktornak számít az, hogy több, különböző paraméterezésű szimulációt is képes egy gombnyomással lekérni.

Mindkét esetben a felhasználónak lehetősége van bizonyos paramétereket megválasztani még a szimulálási fázis előtt:
\begin{itemize}[noitemsep]
	\item kezdeti populáció mérete
	\item defektálók aránya 
	\item generáció szám (szimuláció hossza)
	\item kooperáló sejtek termelési költsége 
	\item diffúziós távolság mérete
	\item akarja-e a felhasználó, hogy a sejtek osztódásra legyenek képesek
\end{itemize}
Ezen paramétereket felhasználva kigenerálható egy Voronoi diagram mely a sejteket ábrázolja. A simulate gombot megnyomva ez az adatcsomag eljut a szerverhez amely elvégzi az erőforrás igényes számításokat melynek eredményét visszaküldi a kliensnek, ami majd azt megjeleníti.

Továbbá a szimulációs oldalon az alábbi paraméterek is változtathatóak:
\begin{itemize}[noitemsep]
	\item a V függvény meredeksége
	\item a V függvény áthajlási pontjának helye
	\item a gradiens alakja
	\item a gradiens meredeksége az áthajlási pontban
\end{itemize}
melyek segítségével a felhasználónak sokkal nagyobb beleszólása van abba, hogy a populáció hogyan is viselkedjen. Mivel ezek a felhasználónak csak puszta számok, elég nyers adat, így segítségére siet két grafikon (\ref{fig:SimulationFunctionDiagrams}), mely a paraméterlista mellett kapott helyet. Ezen grafikonokon a két függvény jelenítődik meg, melyek interaktívak, azaz az értékek változtatásával a grafikon is átalakul, így szemléltetve a felhasználó előtt, hogyan is néz ki az adott V (\ref{eq:payoffGradient}) illetve g (\ref{eq:diffGradient}) függvény.

\begin{figure}[ht!]
	\centering
	\includegraphics[width=90mm]{images/EGTIB.jpg}
	\caption{A V és g függvények alakja adott paraméterekre}
	\label{fig:SimulationFunctionDiagrams}
\end{figure}


\paragraph{Eredmények}- kezdetben már azzal is megelégedtünk, ha egy szimulációt képesek voltunk megjeleníteni, viszont egy idő után azt vettük észre, hogy milyen jó lenne ha ezeket nem kellene mindig újra és újra generálni. Így született meg az eredményeket tartalmazó rész mely megjeleníti az oldalon futtatott szimulációkat egy error bar segítségével. Mivel a bemeneti paraméterek eléggé változatosak lehetnek, így valamilyen szinten ezen adatokat a megjelenítés során szűrni kell. Ezért a felhasználó ezt köteles is megtenni mielőtt az eredményeket megtekintené, az alábbiak szerint:
\begin{itemize}[noitemsep]
	\item defektáló sejtek aránya
	\item kooperálási költség 
	\item interakció távolsága
\end{itemize}
Ezen adatok megadása után, az összes olyan szimuláció mely kielégíti a feltételeket egyesítve lesz egy grafikonon (\ref{fig:SimulationResults}).

\begin{figure}[ht!]
	\centering
	\includegraphics[width=90mm]{images/EGTIB.jpg}
	\caption{Eddigi szimulációk eredményei}
	\label{fig:SimulationResults}
\end{figure}


\section{Felhasznált technológiák}

A szoftver egy szerverből és egy kliensből áll, melyek közötti kommunikáció a már jól megszokott HTTP mellett, websocketen keresztül is folyik. Az utóbbira azért van szükség, mert gyorsítja az adatok áramlását, és alkalmas nagyobb mennyiségű adat átvitelére, mely a szimuláció során keletkezik. 

Kliens oldali technológiák közé sorolandó a Bootstrap \cite{soft:bootstrap}, mely segítségével nem csak desktopon de mobil platformon is elegáns az oldal kinézete, valamint az AngularJS \cite{soft:angular}, ami biztosítja az oldal dinamikusságát. A Paper.js \cite{soft:paper} a rajzok megjelenítéséért felelős, és ennek egy segédkönyvtára \cite{soft:voronoiModule} pedig kifejezetten a voronoi diagram ábrázolásáért, úgy vizuálisan mint adatszerkezeti szinten is. A grafikonok kirajzolását kezdetben a Highchartsra \cite{soft:highcharts} míg a későbbiekben a könnyebben használható Plotly.js-re \cite{soft:plotly} bíztuk.

A szerverünk NodeJS \cite{soft:node} alapú, az Express \cite{soft:express} keretrendszert használja fel. Itt folyik az erőforrás igényes számítások nagy része, így itt is jelen van a Paper.js voronoi modulja \cite{soft:voronoiModule}.

Mindezt összefogva és belerakva egy Dockerbe \cite{soft:docker}, már mehet is egy felhőbe, a mi esetünkben ez a Heroku (https://egtib.herokuapp.com/). A fejlesztési és tesztelési folyamatot amennyire csak lehetett megpróbáltuk automatizálni, így a Travis CI \cite{soft:travis} az, ami a projektünket teszteli és amennyiben szükséges kitelepíti a legújabb verziót a felhőbe.

Az alkalmazásunk minőségét unit és E2E tesztekkel próbáltuk meg biztosítani, ezekhez a Mocha \cite{soft:mocha} és TestCafe \cite{soft:testcafe} keretrendszereket használtuk.

\begin{figure}[ht!]
	\centering
	\includegraphics[width=\linewidth]{images/Architecture}
	\caption{A projekt architektúrája\label{fig:Architecture}}
\end{figure}
