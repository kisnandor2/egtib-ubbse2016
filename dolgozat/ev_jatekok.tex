\chapter{Evolúciós játékok}

\section{Klasszikus játékelméleti fogalmak}

A játékelméletet a matematika egy fő ágának tekintjük, mely olyan helyzetekkel foglalkozik, amelyben legalább két döntéshozó verseny helyzetbe kerül és saját nyereség függvényét próbálja maximalizálni. A nehézség abban rejlik, hogy minden szereplő befolyásolja legalább egy másik szereplő döntését. Ezen \textit{játékok} első elgondolásra társasági játékokra utalnak, de ezen játékokkal modellezhetünk közgazdaságtanban vagy szociológiában, katonai és biológia alkalmazásuk is létezik.

Bizonyos szempontok alapján, ezeket a játékokat, kategorizálni lehet. Itt kiemelhetjük a kooperatív játékot, amelyben a játékosok felismerve azt, hogy hasznot húzhatnak az együttműködésből, valamilyen szinten csoportosulnak. Illetve ennek párját, a nem kooperatív játékot, ahol a döntéshozó játékosok egymásnak versenytársai, bármiféle együttműködés önkéntes módon jöhet létre.

Egy másik osztályozási szempont szerint léteznek szimmetrikus játékok, ahol a haszon csak a választott stratégiától függ, a játékos személyétől nem. Aszimmetrikus játékról akkor beszélünk, ha a játékosok nem cserélhetőek fel anélkül, hogy a stratégiák nyereségén változtatnánk.

\subsection{Statikus és szekvenciális játékok}
Minden játék esetén szükségünk van az idő fogalmára, így eszerint két típusúról beszélhetünk. A statikus játékok esetében a döntéshozók már a játék legelején, egymástól függetlenül döntenek. Ezzel szemben, a dinamikus játékok esetében számít a lépések sorrendje, mivel minden döntéshozó ismeri az a többiek előző lépéseit. Amennyiben egy statikus játékot ismételten játszanak, úgy azt a dinamikus kategóriába sorolhatjuk, hiszen egy kör elején az eddigi körök alapján hozza meg döntését egy játékos. Ezen két típusú játék ábrázolása is különbözik egymástól. A statikusokat általában egy mátrix segítségével írjuk le, amelyről leolvasható az összes lehetséges kimenet. Míg a dinamikusok esetében egy véges irányított fa ábrázolja az egymás utáni lépéseket, amely gyökere a kezdőállapot, ahonnan bármely ponthoz egyetlen egy út vezet.

\subsection{Nash-egyensúly}
Amennyiben a játék során létrejön egy olyan állapot, ahol egyik döntéshozónak sem éri meg eltérni a saját alkalmazott stratégiájától, egyik fél sem akar változtatni, akkor az úgynevezett Nash-egyensúlyról beszélünk. Tekintsük \(G = ((N,S_i,u_i), i = 1,...,n)\) rendszert egy véges stratégia játék reprezentációjának, ahol \textit{N} a játékosok halmaza, \textit{n} pedig a játékosok száma. Minden \(i \in N\) játékos a \(S_i\) lépés-halmazból választhat. \(S = S_1 \times S_2 \times ... \times S_n\)  a játék összes kimenetelének halmaza, ahol \(s \in S\) egy stratégia-együttes. Minden \(i \in N\) játékos nyereségét a \(u_i:S \to \mathbb{R}\) hasznosságfüggvény írja le. Jelöljük \((s_i^*,s_i) = (s^*_1,...,s_i,...,s_n^*)\) -vel azt a stratégia-együttest, amelyet az \(s^*\)-ból kapunk ha az \textit{i} játékos stratégiáját kicseréljük \(s_i\)-re. Egy \(s^*\) stratégia-együttes Nash-e\-gyen\-súly\-ban van, ha az \(u_i(s_i^*,s_i)\leq u_i(s^*) \) egyenlőtlenség igaz \(\forall i = 1,...,n, \forall s_i \in S_i, s_i \ne s_i^*\) esetén \cite{nash1951non}.


\section{Játékelmélet a biológiában}
Az evolúciós játékelmélet a játékelmélet biológiai alkalmazása során jött létre, a fejlődő populációk vizsgálata a biológiában. Maynard Smith volt az aki felismerte, hogy a játékosoknak nem feltétlenül kell racionálisan viselkedniük, már az is elegendő ha van egy stratégiájuk. A stratégiák genetikailag öröklődnek, ezek befolyásolják egy egyed jellegét. Attól függően, hogy egy stratégia milyen gyakori illetve milyen más stratégiák vannak még jelen a játékban, meghatározható egy adott stratégia sikeressége. A résztvevők szelekciós sikerét a \textit{fitnesz} fejezi ki (a szaporodásra való relatív esély). Egy stratégiát evolúciósan stabilnak nevezünk, ha az azt alkalmazó populáció nem győzhető le semmilyen más, kezdetben kevés létszámú stratégiával. Az evolúciós játékelméletben az egyensúlyt az evolúciósan stabil stratégia jelenti.

Például egy egyszerű evolúciós játék a következő. Tegyük fel, hogy adott egy bogár populáció, amely ételért verseng (\cite{book:egt}). Az egyszerűség kedvéért csak két stratégia lesz, a bogarak vagy kicsi- vagy nagytestűek. Nem meglepő, hogy a nagyobb testű bogarak több ételhez jutnak hozzá ha kisebb társaikkal versenyeznek. Ha két bogár egymással versenyez, akkor a következő kimenetelek lehetségesek:
\begin{itemize}[noitemsep]
	\item ha két egyforma méretű bogár verseng, akkor egyformán osztják el az ételt
	\item ha egy kicsi és egy nagy verseng, akkor természetes, hogy a nagyobbiknak sokkal több jut
	\item mindkét esetben a nagytestűek az étel egy részét elveszítek, ezt a versengés emészti fel 
\end{itemize}
A játékot legkönnyebben egy payoff mátrixszal lehet leírni ahogyan azt a \ref{fig:beetle} táblázat teszi. Jól látható, hogy a táblázatban szereplő értékek megfelelnek a fenti szabályoknak. Az is látható, hogy ha két nagy bogár verseng, akkor mindkettejüknek sokba kerül a verseny.
Ekkor látható a legjobban, hogy míg a hagyományos játékelméletben a döntéshozók stratégiát választhatnak, addig itt egy egyed örökölte a stratégiáját. Egy nagytestű bogár nem dönthet úgy, hogy ő most kicsi testtel szeretne versengeni, erre nincs lehetősége. Itt a Nash-egyensúlynak már kevesebb értelme van, ezért az lecserélődik evolúciósan stabil stratégiára mint egyensúlyi állapot.

\begin{table}[htb!]
	\centering
	\begin{tabular}{ccccc}
		&       & \multicolumn{2}{c}{Bogár2} &  \\
		\multicolumn{1}{c}{}    &       & Kicsi        & Nagy        &  \\ \cline{3-4}
		\multirow{2}{*}{Bogár1} & Kicsi & \multicolumn{1}{|c|}{5,5}          & \multicolumn{1}{c|}{1,8}         &  \\ \cline{3-4}
		& Nagy  & \multicolumn{1}{|c|}{8,1}          & \multicolumn{1}{c|}{3,3}         & \\ \cline{3-4} \\
	\end{tabular}
	\caption{Bogár játék testméret alapján}
	\label{fig:beetle}
\end{table}