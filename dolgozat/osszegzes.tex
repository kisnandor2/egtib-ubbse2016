\section{Összegzés}

Elmondhatjuk, hogy sikerült egy olyan szoftvert létrehoznunk mely segítségével jobban megérhetjük, hogy mi is történik egy daganaton belül. Fő erőssége a kísérletezésben rejlik, de kutatási vagy akár tanítási célokra is fel lehet használni.

Természetesen hiányosságai is vannak, melyeket a jövőben szeretnénk pótolni. Rendkívül hasznos lenne azon funkció mely során egy terápiát alkalmazunk, vagy ellenanyagokat juttatunk be, melyek a termelést, annak költségét befolyásolják. Komoly kihívásnak számít az, hogy különböző ráktípusokra meghatározzuk az őket leíró paramétereket. A számítások hatékonyságágának növelése is egy fontos feladat, ami maga után vonná azt a tényt, hogy nagyobb populációkkal is szimulálhatunk. 

Az általunk végzett szimulációk arra a feltételezésre engednek következtetni, hogy a sejtek játéka bizonyos határokon belül leírja a daganatok viselkedését és hisszük azt, hogy a játékelmélettel ezen területet ki lehet aknázni.