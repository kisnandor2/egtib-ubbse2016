\chapter{Bevezető}

A rákkutatás a mai napig egy nagyon aktív kutatási terület, melyben bizonyos matematikai modelleket is jól lehet alkalmazni. Ilyen például az evolúciós játékelmélet is, aminek a biológiai alkalmazása még egészen újnak számít. Dolgozatunk alapjait ezen területen írt legfrissebb munkák szolgálják. Az evolúciós játékelmélet segítségével a következőkben daganatos sejteket fogunk  modellezni és azok viselkedését kielemezni.

Dolgozatunkat több nagyobb részre osztjuk. Elsősorban szükségünk lesz néhány játékelméleti fogalomra, melyek nélkül a dolgozatot nem tekinthetnénk teljesnek. Itt kerülnek majd bevezetésre a játékok, ezek típusai és megismerkedünk az evolúciós játékelmélet alapjaival. Itt megemlítjük a fajon belüli rivalizálást és egy egyszerű példát az evolúciós játékra. Minket jelen esetben nem az egyedek viselkedése érdekel, hanem az, ami sejtek szintjén történik.

A következő fejezetben röviden részletezzük a daganatos sejteket és azok viselkedését, valamint az itt alkalmazható játékelméleti modelleket. Bemutatjuk az általunk használt modellt, szereplőit és játékszabályait, valamint azt, hogy milyen újításokat tettünk hozzá az eredetihez képest. Azt is tárgyaljuk, hogy mik azok a Voronoi diagramok és miért pont azt használjuk.

Mindezt a sok elméleti fogalmat egy alkalmazásba csomagoltuk, így a következő fejezet bemutatja az alkalmazásunkat, amit egy webes felület képez, amelyen a fenti modellek jeleníthetőek meg, a sejtek viselkedése, a populáció időbeli alakulása és ezek statisztikája. Részletezzük az alkalmazás funkcionalitásait és felépítését, valamint a felhasznált technológiákról is beszámolunk.

Végül pedig összegezzük a szimulációk eredményét. Megvizsgáljuk, hogy az alkalmazás képes-e reprodukálni az eredeti modellt. Megnézzük, hogy a paraméterek változása hogyan befolyásolja a játék végkimenetelét. Legvégül pedig összevetjük az eredeti modellt a saját kibővített változatunkkal.