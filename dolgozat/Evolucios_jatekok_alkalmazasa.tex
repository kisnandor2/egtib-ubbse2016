\documentclass[10pt,a4paper]{article}
\usepackage[utf8]{inputenc}
\usepackage{amsmath}
\usepackage{amsfonts}
\usepackage{amssymb}
\usepackage{graphicx}
\title{Evolúciós játékok alkalmazása daganatos sejtek modellezésében}
\begin{document}

\maketitle

\begin{abstract}
	Dolgozatunkban az evolúciós játékok alkalmazását mutatjuk be a sejtbiológiában. Már a sejtek szintjén megfigyelhetőek kooperatív illetve kompetitív viszonyok, amelyek játékelmélettel modellezhetőek. Ezeket, az alkalmazásunk elméleti háttereként is szolgáló matematikai modelleket használják a rákkutatásban is, ahol a sejtek mint játékosok vannak jelen. Fő célunk játékelméleti modellek kiterjesztése a sejtek interakcióinak vizsgálatára. A projekt gyakorlati részét egy webes felület képezi, amin megjeleníthető a különböző sejtpopulációk viselkedése, időbeni alakulása. A szimuláció paramétereit, a generációk számát a felhasználó módosíthatja, ezáltal valós időben betekintést nyerve a daganatok dinamikájába.
\end{abstract} 

{ \baselineskip 1ex
	\parskip 1ex
	\tableofcontents
}

\chapter{Bevezető}

A rákkutatás a mai napig egy nagyon aktív kutatási terület, melyben bizonyos matematikai modelleket is jól lehet alkalmazni. Ilyen például az evolúciós játékelmélet is, aminek a biológiai alkalmazása még egészen újnak számít. Dolgozatunk alapjait ezen területen írt legfrissebb munkák szolgálják. Az evolúciós játékelmélet segítségével a következőkben daganatos sejteket fogunk  modellezni és azok viselkedését kielemezni.

Dolgozatunkat több nagyobb részre osztjuk. Elsősorban szükségünk lesz néhány játékelméleti fogalomra, melyek nélkül a dolgozatot nem tekinthetnénk teljesnek. Itt kerülnek majd bevezetésre a játékok, ezek típusai és megismerkedünk az evolúciós játékelmélet alapjaival. Itt megemlítjük a fajon belüli rivalizálást és egy egyszerű példát az evolúciós játékra.

A következő fejezetben röviden részletezzük a daganatos sejteket és azok viselkedését, valamint az itt alkalmazható játékelméleti modelleket. Bemutatjuk az általunk használt modelleket, szereplőit és játékszabályait, valamint azt, hogy milyen újításokat tettünk hozzá az eredetihez képest. Azt is tárgyaljuk, hogy mik azok a Voronoi diagramok és miért pont azt használjuk.

Mindezt a sok elméleti fogalmat egy alkalmazásba csomagoltuk, így a következő fejezet bemutatja az alkalmazásunkat, amit egy webes felület képez, amelyen a fenti modellek jeleníthetőek meg, a sejtek viselkedése, a populáció időbeli alakulása és ezek statisztikája. Részletezzük az alkalmazás funkcionalitásait és felépítését, valamint a felhasznált technológiákról is beszámolunk.

Végül pedig összegezzük a szimulációk eredményét. Megvizsgáljuk, hogy az alkalmazás képes-e reprodukálni az eredeti modellt. Megnézzük, hogy a paraméterek változása hogyan befolyásolja a játék végkimenetelét és összevetjük az eredeti modellt a saját kibővített változatunkkal.

Itt megemlítenénk azt is, hogy ezen dolgozat egy kezdetlegesebb formája már bemutatásra került a XX. Reál- és Humántudományi Erdélyi Tudományos Diákköri Konferencián, ahol második díjjal lett jutalmazva.
\section{Evolúciós játékok}


\subsection{Összehasonlítás a klasszikus játékelmélettel}
\subsection{Evolúciós stabil stratégia -- fitnesz mint nyereség}
\subsection{Játékok dinamikája(Dynamic game theory)}
\subsubsection{Replikátor dinamika}
\subsubsection{Adaptív dinamika (Adaptive dynamics)}
\subsection{Térbeli játékok (spatial games)}


\section{Daganatos sejtek modellezése}
A rosszindulatú daganatok sokszoros mutáció következtében létrejött szabályozatlan, elburjánzó sejttömörülések, amelyek az egészséges szövetbe betörve versenyeznek az erőforrásokért. Egy daganaton belül többféle daganatos sejt is megtalálható, mindegyik különböző módon hatva a környezetében található sejtekre. Ez a heterogenitás segíti a burjánzást, az egyre agresszívabb sejttípusok megjelenését és az erősebb ellenállást a kezelésekre. A játékelmélet egy lehetséges eszköz lehet a különböző típusok közötti versengés valamint az alkalmazott stratégiák modellezésére.

\subsection{Lehetséges modellek}
(pontos ref. és konkrét példák következik)
A játékelméleti modellek segítségével többféleképpen is megközelíthető a probléma. Játékosok lehetnek csak a daganatos sejtek és a játékot az erőforrásokért folytatott \textit{verseny} jelenti. Ez a verseny átalakulhat \textit{parazitizmussá} ha a bizonyos növekedési faktorokat termelő egészséges sejteket tekintjük az egyik félnek, és az ezt kihasználó daganatos sejteket a másiknak. Az immunsejtek játszhatják a \textit{ragadozó} szerepét de felléphet \textit{mutualizmus} is például az erek belső rétegének sejtjei és a daganatos sejtek között. 
(Ide jön több példa, cikkekből idézések stb.)
\subsubsection{...}
\subsubsection{Public goods game}
A daganatos sejtek több jellegzetességének kialakulásában fontos szerepet játszanak az általuk termelt növekedési faktorok és jelzőmolekulák. Ezek segítenek többek között abban, hogy a rákos sejtek a saját növekedésüket ösztönözzék, hogy képesek legyenek kijátszani a szervezet védekezési rendszerét és hogy szétterjedjenek távoli szövetekbe (metasztázis). Mivel ezek az anyagok kikerülnek a sejtek közötti térbe, nem csak az őket termelő sejtekre vannak hatással, hanem az ezeket körülvevőkre is. Így a növekedési faktor felfogható mint egy "közös jó" (\textit{public good}). A evolúciós játékelmélet megfelelő keretet biztosít, mivel nem feltételez racionális viselkedést és egy stratégia sikerességét megadja a populáción belüli gyakorisága\cite{archetti2016cooperation}.

\subsection{A játék felépítése}
\subsubsection{Voronoi diagram}
Az általunk vizsgált modell eltér a megszokottól abban, hogy a sejtek ábrázolására voronoi hálózatokat használ. Térbeli játékok esetén általában szabályos pontrácsokkal (??) vagy skála-független hálózatokkal dolgoznak. Míg az előbbi figyelmen kívül hagyja a kapcsolatok sokféleségét, az utóbbi nem megfelelő, ha az egyedek egy síkban helyezkednek el. (?? kérdéses rész)

Voronoi meghatározás  

\subsubsection{A játék szereplői és a stratégiák}
A játékban résztvevő sejtek két stratégia közül választhatnak: \textit{kooperálnak}, azaz termelnek növekedési faktorokat, vagy \textit{defektálnak}, azaz nem vesznek részt a faktorok termelésében. A stratégiák \textit{nyereségének (payoff)} kiszámítására a következő képletet használtuk: \(p = b(j) - c\) ahol a \textit{c} a növekedési faktor előállításának költsége, \textit{j} a csoportban résztvevő kooperatív sejtek száma és
\begin{equation}
b(j) = [V(j) - V(0)]/[V(n) - V(0)]
\end{equation}
a
\begin{equation}
V(j) = 1/[1 + e^{(-s(j-k)/n)}]
\end{equation}
függvény normalizált alakja. A \textit{k} befolyásolja az áthajlási pont helyét, az \textit{s} irányítja a függvény meredekségét az áthajlási pontban. A vizsgált csoport méretét az \textit{n} jelöli. 

A kezdeti populációban véletlenszerűen elhelyezünk defektáló sejteket (alapértelmezetten az arányuk 0.05). Ezután minden körben véletlenszerűen kiválasztunk egy sejtet és annak egy szomszédját és megvizsgáljuk a nyereségeket. Amennyiben a szomszéd stratégiája kifizetődőbb, a kiválasztott sejt átveszi azt. Minden kör végén aktualizáljuk a nyereségeket (hiszen a stratégiák változtatásával változik a \textit{j} értéke és azzal a nyereségek is). 

\subsubsection{Diffúziós távolság}
A dolgozatunkban vizsgált modell másik újítása, hogy nem csak az elsőfokú szomszédokat veszi figyelembe, hanem egy bizonyos diffúziós távolságon belül található összes sejtet. A szomszédos termelő sejtek befolyásának súlyozására egy diffúziós gradienst vezetünk be. 
\begin{equation}
G(i) = [g(i) - g(0)]/[g(D) - g(0)] 
\end{equation}
\begin{equation}
g(i) = 1/[1 + e^{(-z(i-d)/D)}]
\end{equation}
függvények segítségével kiszámolhatjuk az \textit{i} távolságra található kooperatív sejtek befolyásának mértékét. Így a termelők száma nem a fent említett \textit{j} lesz, hanem a \textit{G}-vel kiszámított súlyozott összeg. A képletben szereplő \textit{d} és \textit{D} a diffúziós gradiens alakját határozzák meg, a \textit{z} a gradiens meredekségét az áthajlási pontban\cite{archetti2016cooperation}.

\subsection{Osztódás}
Kezdeti szimulációink során nem vettük figyelembe azt, hogy a sejtek életük során szaporodni, azaz osztódni is szoktak. A legtöbbet használt modell, mely elég közel áll a természethez, az a Gompertz modell, amely leírja a tumor méretének időbeli változását:
\begin{equation}
	n_t = K \bigg(\frac{n_0}{K} \bigg) ^ {e^{(- \alpha t)}}
\end{equation}

\newcommand{\projectName}{Az EGTIB}

\section{\projectName{} bemutatása}

\projectName{} projekt célja egy olyan felhasználóbarát felület, mely lehetőséget teremt a daganatos sejtek játékelméleti modellezésére és szimulációjára, valamint a szimulációs eredmények megjelenítésére. Ezért született meg, a mai trendeket figyelembe véve, egy kliens-szerver architektúrán alapuló webalkalmazás, mely részben a\cite{archetti2016cooperation}-ben megjelent modellt implementálja, és ahhoz új funkcionalitásokat is hozzáad.

\begin{figure}[ht!]
	\centering
	\includegraphics[width=90mm]{images/EGTIB.jpg}
	\caption{Pillanatkép az alkalmazásról \label{fig:SimulateWithDiagram}}
\end{figure}

\subsection{Funkcionalitások}

Megfelelve a követelményeknek melyeket a "megrendelő" állított fel, a felhasználónak lehetősége van bizonyos paramétereket megválasztani még a szimulálási fázis előtt:
\begin{itemize}[noitemsep]
	\item kezdeti populáció mérete
	\item defektálók aránya 
	\item generáció szám (szimuláció hossza)
	\item kooperáló sejtek termelési költsége 
	\item diffúziós távolság mérete
	\item akarja-e a felhasználó, hogy a sejtek osztódásra legyenek képesek
\end{itemize}

Az alkalmazásunkat felhasználási szempontból két fő komponensre oszthatjuk:

\paragraph{Vizualizáció}- képes szimulálni kevés sejtből (max. 500) álló populációkat és ezeket generációnként meg is tudja jeleníteni. A megjelenítés interaktív, ami azt jelenti, hogy bármikor meg lehet állítani, folytatni vagy akár tekergetni mint egy filmet. Továbbá mindehhez tartozik egy grafikon is, mely a populációnak változását ábrázolja generációnként.

\paragraph{Szimuláció}- alkalmas nagyobb (max. 2000) populációval dolgozni, viszont ezek megjelenítése már túl sok időt venne igénybe, így csak a fentebb említett grafikon segítségével ábrázolja azt, hogy mi is történik a sejtek között. Előnye a vizualizációs részhez képest elsősorban az, hogy sokkal nagyobb populációval tud dolgozni, de a paraméterlista is változatosabb (a szimuláció során használt függvények viselkedésébe is bele tud szólni a felhasználó) és hatalmas kényelmi faktornak számít az, hogy több, különböző paraméterezésű szimulációt is képes egy gombnyomással lekérni.

\subsection{Felhasznált technológiák}

A szoftver egy szerverből és egy kliensből áll, melyek közötti kommunikáció a már jól megszokott HTTP mellett, websocketen keresztül is folyik. Az utóbbira azért van szükség, mert gyorsítja az adatok áramlását, és alkalmas nagyobb mennyiségű adat átvitelére, mely a szimuláció során keletkezik. 

Kliens oldali technológiák közé sorolandó a Bootstrap, mely segítségével nem csak desktopon de mobil platformon is elegáns az oldal kinézete, valamint az AngularJS, ami biztosítja az oldal dinamikusságát. A Paper.js a rajzok megjelenítéséért felelős, és ennek egy segédkönyvtára pedig kifejezetten a voronoi diagram ábrázolásáért, úgy vizuálisan mint adatszerkezeti szinten is. A grafikonok kirajzolását kezdetben a Highchartsra míg a későbbiekben a könnyebben használható Plotly.js-re bíztuk.

A szerverünk NodeJS alapú, az Express keretrendszert használja fel.Itt folyik az erőforrás igényes számítások nagy része, így itt is jelen van a Paper.js voronoi modulja.

Mindezt összefogva és belerakva egy Dockerbe, már mehet is egy felhőbe, a mi esetünkben ez a Heroku (https://egtib.herokuapp.com/). A fejlesztési és tesztelési folyamatot amennyire csak lehetett megpróbáltuk automatizálni, így a Travis CI az, ami a projektünket teszteli és amennyiben szükséges kitelepíti a legújabb verziót a felhőbe.

Az alkalmazásunk minőségét unit és E2E tesztekkel próbáltuk meg biztosítani, ezekhez a Mocha és TestCafe keretrendszereket használtuk.

\begin{figure}[ht!]
	\centering
	\includegraphics[width=\linewidth]{images/Architecture}
	\caption{A projekt architektúrája\label{fig:Architecture}}
\end{figure}

\section{A szimuláció eredményei}
\subsection{A költség és nyereség hatása}
\subsection{Diffusion gradient hatása}
\subsection{Összegzés}


\bibliography{articles}
\bibliographystyle{ieeetr}
\end{document}