\section{Az EGTIB bemutatása}
\begin{frame}
	\frametitle{Az EGTIB bemutatása}
	\begin{block}{Az EGTIB projekt célja}
		\begin{itemize}
			\item felhasználóbarát felület
			\item lehetőség a daganatos sejtek modellezésére 
			\item szimulációk megjelenítése
			\pause
			\item a szimulációs eredmények megjelenítésére
		\end{itemize}
	\end{block}

	\begin{figure}[ht!]
		\centering
		\includegraphics[width=0.6\linewidth]{images/voronoi_page.png}
		\caption{Pillanatkép az alkalmazásról}
		\label{fig:SimulateWithDiagram}
	\end{figure}
\end{frame}

\subsection{Funkcionalitások}
\begin{frame}
	\frametitle{Funkcionalitások}
	\begin{block}{Válaszható paraméterek}
		\begin{itemize}
			\item kezdeti populáció mérete
			\item defektálók aránya 
			\item generáció szám (szimuláció hossza)
			\item kooperáló sejtek termelési költsége 
			\item diffúziós távolság mérete
			\item legyenek a sejtek osztódásra képesek?
		\end{itemize}
	\end{block}

	\begin{block}{}
		\begin{columns}[T]
			\begin{column}{0.5\linewidth}
				\begin{equation}
				V(j) = 1/[1 + e^{(-s(j-k)/n)}]
				\end{equation}
				\begin{itemize}
					\vspace{-1cm}
					\item k - az áthajlási pont helye
					\item s - a függvény meredeksége a áthajlási pontban
					\item n - a csoport mérete
				\end{itemize}
			\end{column}
			\begin{column}{0.5\linewidth}
				\begin{equation}
				g(i) = 1/[1 + e^{(-z(i-d)/D)}]
				\end{equation}
			\begin{itemize}
				\vspace{-1cm}
				\item z - a függvény meredeksége a áthajlási pontban
				\item d és D - a diffúziós gradiens alakja
			\end{itemize}
			\end{column}
		\end{columns}
	\end{block}
\end{frame}

\begin{frame}
	\frametitle{Demó}
	\Huge{\centerline{Demó}}
\end{frame}

\subsection{Felhasznált technológiák}
\begin{frame}
	\frametitle{Felhasznált technológiák}
	\begin{figure}[ht!]
		\centering
		\includegraphics[width=\linewidth]{images/technologies}
	\end{figure}
\end{frame}
