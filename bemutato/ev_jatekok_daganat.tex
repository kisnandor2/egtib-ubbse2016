\section{Daganatos sejtek modellezése}
\begin{frame}
\frametitle{Daganatos sejtek modellezése}
A rosszindulatú daganatok sokszoros mutáció következtében létrejött szabályozatlan, elburjánzó sejttömörülések, amelyek az egészséges szövetbe betörve versenyeznek az erőforrásokért. Egy daganaton belül többféle daganatos sejt is megtalálható, mindegyik különböző módon hatva a környezetében található sejtekre. Ez a heterogenitás segíti a burjánzást, az egyre agresszívabb sejttípusok megjelenését és az erősebb ellenállást a kezelésekre. A játékelmélet egy lehetséges eszköz lehet a különböző típusok közötti versengés valamint az alkalmazott stratégiák modellezésére.

A evolúciós játékelméleti modellek segítségével többféleképpen is megközelíthető a probléma. Játékosok lehetnek csak a daganatos sejtek és a játékot az erőforrásokért folytatott \textit{verseny} jelenti. Ez a verseny átalakulhat \textit{parazitizmussá} ha a bizonyos növekedési faktorokat termelő egészséges sejteket tekintjük az egyik félnek, és az ezt kihasználó daganatos sejteket a másiknak. Az immunsejtek játszhatják a \textit{ragadozó} szerepét de felléphet \textit{mutualizmus} is például az erek belső rétegének sejtjei és a daganatos sejtek között. 
\end{frame}

\subsection{Közjó játék (Public goods game)}
\begin{frame}
\frametitle{Közjó játék (Public goods game)}
A daganatos sejtek több jellegzetességének kialakulásában fontos szerepet játszanak az általuk termelt növekedési faktorok és jelzőmolekulák. Ezek segítenek többek között abban, hogy a rákos sejtek a saját növekedésüket ösztönözzék, hogy képesek legyenek kijátszani a szervezet védekezési rendszerét és hogy szétterjedjenek távoli szövetekbe (metasztázis). Mivel ezek az anyagok kikerülnek a sejtek közötti térbe, nem csak az őket termelő sejtekre vannak hatással, hanem az ezeket körülvevőkre is. Így a növekedési faktor felfogható mint egy "közös jó" (\textit{public good}). Az evolúciós játékelmélet megfelelő keretet biztosít, mivel nem feltételez racionális viselkedést és egy stratégia sikerességét megadja a populáción belüli gyakorisága \cite{archetti2016cooperation}.
\end{frame}

\subsection{A játék felépítése}
\subsubsection{Voronoi diagram}
\begin{frame}
	\frametitle{Voronoi diagram}
	\begin{block}{}
		\begin{itemize}
			\item a sejtek ábrázolása Voronoi hálózatokkal
			\item szabályos pontrács - figyelmen kívül hagyja a kapcsolatok sokféleségét\cite{archetti2016cooperation}
			\item skála-független hálózatok - nem megfelelő, ha az egyedek egy síkban helyezkednek el\cite{archetti2016cooperation}
		\end{itemize}
	\end{block}
	
	\centering
	\includegraphics[width=0.5\linewidth]{images/Voronoi}
\end{frame}

\subsubsection{A játék szereplői és a stratégiák}
\begin{frame}
	\frametitle{A játék szereplői és a stratégiák}
	A sejtek:
	\begin{itemize}
		\item \textit{kooperálnak} - termelnek növekedési faktorokat
		\item \textit{defektálnak} - nem vesznek részt a faktorok termelésében
	\end{itemize}
	A stratégiák \textit{nyereségének (payoff)} kiszámítása: \(p = b(j) - c\)
	\begin{itemize}
		\item \textit{c} - a növekedési faktor előállításának költsége
		\item \textit{j} - a csoportban résztvevő kooperatív sejtek száma
	\end{itemize}
	ahol 
	\begin{equation}
		b(j) = [V(j) - V(0)]/[V(n) - V(0)]
	\end{equation}
	\begin{equation}
		\label{eq:payoffGradient}
		V(j) = 1/[1 + e^{(-s(j-k)/n)}]
	\end{equation}
\end{frame}

\begin{frame}
	\frametitle{Szabályok}
	\begin{itemize}
		\item véletlenszerűen elhelyezünk defektáló sejteket (arányuk 0.05)
		\item minden körben véletlenszerűen kiválasztunk egy sejtet és annak egy szomszédját 
		\item megvizsgáljuk a nyereségeket
		\item ha a szomszéd stratégiája kifizetődőbb, a kiválasztott sejt átveszi azt
		\item minden kör végén aktualizáljuk a nyereségeket(aszinkron módon számolunk)
	\end{itemize}
\end{frame}

\subsubsection{Diffúziós távolság}
\begin{frame}
	\frametitle{Diffúziós távolság}
	\begin{block}{}
		Nem csak az elsőfokú szomszédokat vesszük figyelembe, hanem egy bizonyos diffúziós távolságon belül található összes sejtet. A szomszédos termelő sejtek befolyásának súlyozására egy diffúziós gradienst vezetünk be:
		\begin{equation}
			G(i) = [g(i) - g(0)]/[g(D) - g(0)] 
		\end{equation}
		\begin{equation}
			\label{eq:diffGradient}
			g(i) = 1/[1 + e^{(-z(i-d)/D)}]
		\end{equation}
	\end{block}
	\begin{block}{}
		A termelők száma a \textit{G}-vel kiszámított súlyozott összeg.\\
		A sejtek nem csak a közvetlen szomszédokkal vannak kölcsönhatásban, hanem távolabbiakkal is.
	\end{block}
\end{frame}

\subsection{Osztódás}
\begin{frame}
	\frametitle{Osztódás}
	\begin{block}{A Gompertz modell - a tumor méretének időbeli változása}
		\begin{equation}
			n_t = K \bigg(\frac{n_0}{K} \bigg) ^ {e^{(- \alpha t)}},
		\end{equation}
		\begin{itemize}
			\item $n_t$ - a populáció mérete a $t$ időpillanatban
			\item $n_0$ - a populáció kezdeti mérete
			\item $K$ - az elérhető maximális mérete a tumornak
			\item $\alpha$ - egy konstans, a sejtek burjánzási képességével áll összefüggésben
		\end{itemize}
	\end{block}
	\pause
	\begin{block}{}
		\begin{itemize}
			\item az eddigi modellbe könnyen beépíthető
			\item nem veszi figyelembe, hogy éppen milyen típusú sejt (kooperáló/defektáló) osztódik
			\item nem minden esetben jó ez a megközelítés
		\end{itemize}
	\end{block}
\end{frame}